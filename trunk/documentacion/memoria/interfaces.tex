\chapter{Manual de modificaci�n de interfaces}
Una de las grandes ventajas que presenta Tapestry 5 radica en la posibilidad de realizar dise�o de la interfaz sin necesidad de tener conocimiento de como se implementa un markup. Esto permite que un dise�ador gr�fico con conocimientos de html puro sea capaz de realizar cambios a su gusto. En este ap�ndice se mostrar� la forma de realizar estos cambios de manera sencilla, de modo que sirva a modo de manual para su modificaci�n.
Antes de nada, cabe destacar que la modificaciones que se pueden realizar sobre la vista se pueden realizar de dos forma espec�ficas:
\begin{itemize}
	\item Modificando el CSS asociado a las diferentes p�ginas, pero que obliga a que la estructura de la vista no sea modificable.
	\item Modificando las plantillas o tml asociados a cada p�gina dentro de la implementaci�n, lo cu�l si permite modificar la estructura de la vista.
\end{itemize}
Como el primer punto no supone ning�n problema, ya que s�lo se necesita mo\-dificar el fichero css determinado, no vamos a centra en como se puede modificar un tml.

\begin{figure}[htp]
\centering
\includegraphics*[width=15cm]{imagenes/votetml.png}
\caption{Ejemplo p�gina tml}
\label{tml}
\end{figure}

En esta figura se muestran tres tipos de c�digo representados en colores dife\-rentes:
\begin{itemize}
	\item En negro se recoge el c�digo html puro.
	\item En rojo se distingue el c�digo perteneciente a componentes embebidos.
	\item En azul est�n los mensajes internacionalizados y los objetos inyectados dentro de la p�gina.
\end{itemize}

Dentro de nuestro ejemplo podemos distinguir:
\begin{itemize}
	\item \textbf{Declaraci�n de un espacio de nombres}
	{\tt \scriptsize
\begin{verbatim}
<html xmlns="http://www.w3.org/1999/xhtml"
xmlns:t="http://tapestry.apache.org/schema/tapestry_5_0_0.xsd">
\end{verbatim}
}
	\begin{itemize}
		\item Cualquier elemento o atributo que tenga prefijo t pertenece al espacio de nombres de Tapestry
		\item En el markup generado, no se incluye la importaci�n del espacio de nombres de Tapestry
	\end{itemize}
	\item \textbf{Componentes}
	
	Es posible incluir componentes dentro de una plantilla utilizando el elemento correspondiente (dentro del espacio de nombres de Tapestry).
{\tt \scriptsize
\begin{verbatim}
<t:if t:test="userSessionExists">
</t:if>
\end{verbatim}
}
	\item \textbf{Mensajes internacionalizados}
{\tt \scriptsize
\begin{verbatim}
<spam class="favourite" title="${message:markAsFavourite}" />
\end{verbatim}
}

	\textit{Message}: indica que valor es la clave de un mensaje del cat�logo de mensajes. Dicho cat�logo forma parte del soporte de internacionalizaci�n que proporciona Tapestry:
\begin{itemize}
	\item Permiten que los textos fijos de las plantillas, en lugar de escribirlos directamente en ellas, se escriban
en un fichero de mensajes
	\item Ficheros .properties
	\item Existe un catalogo de mensajes general para toda la aplicaci�n
	\item Es posible definir un fichero de mensajes espec�ficos para una p�gina
\end{itemize}
	\item \textbf{Objetos inyectados o propiedades}
	
	En el caso de acceso a propiedades es posible:	
\begin{itemize}
	\item Referirse a subpropiedades
	\item Invocar a un m�todo del objeto que no sea un getter (utilizando pa\-rentesis a continuacion del
nombre del metodo)
\end{itemize}
\end{itemize}

Gracias a todo esto que se acaba de explicar, un dise�ador podr�a modificar la estructura y la apariencia de la vista teniendo en cuenta que partes del c�digo deben permanecer inalteradas y cuales son modificables, en este caso, el html puro dentro del \textit{tml}.
