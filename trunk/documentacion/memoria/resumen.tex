\section*{Resumen}
El proyecto \textit{El mas} es un concurso-reality multiplataforma (televisi�n, Web, m�vil), en el que los concursantes env�an un video de hasta 90 segundos cada semana. Esos videos son votados mediante una plataforma Web y los que obtienen m�s votos son emitidos en un programa de televisi�n semanal de 30 minutos de duraci�n. 


En el presente proyecto se acomete el dise�o e implementaci�n de un portal web destinado a albergar el contenido multimedia de un espacio televisivo. A su vez �ste debe servir como fuente de contenidos. Para facilitar la difusi�n del programa se realizar� una aplicaci�n web que permita su manejo tanto en navegadores convencionales como en navegadores utilizados por dispositivos m�viles.


La aplicaci�n resultante de este proyecto conformar� parte de un proyecto conjunto entre la Facultad de Inform�tica y la Facultad de Ciencias de la Comunicaci�n de la Universidad de la Coru�a. De este modo, el sistema tendr� que ser probado adecuadamente, ya que se presume su posible comercia\-lizaci�n.


La aplicaci�n desarrollada permite automatizar labores de administraci�n (comprobaci�n de videos, recodificaci�n, gesti�n de abusos, elaboraci�n de listas de videos m�s votados para elaborar el programa de tv) y da soporte completo a la votaci�n e incorporaci�n de videos. Tambi�n permite una serie de utilidades como realizar comentarios, notificar abusos, etc. La aplicaci�n emplea tecnolog�as libres y avanzadas como Hibernate, Spring y Tapestry.
Se ha implementar� un API completo que sirva de enlace entre la aplicaci�n
web y un sistema de compresi�n de video instalado en el propio servidor de
la aplicaci�n.


Adem�s de una descripci�n de los detalles de implementaci�n, el proceso de desarrollo, la metodolog�a seguida y las caracter�sticas finales del producto, se exponen en esta memoria los resultados de las pruebas de unidad, integraci�n y estr�s. Se proporcionan los manuales necesarios para instalar y adaptar el producto.

\vspace{2cm}

\textbf{Palabras claves: } Hibernate 3, Spring Framework, Tapestry 5, FLV Media Player, CSS, Javascript, Junit, Selenium, JMeter, FFMpeg.
