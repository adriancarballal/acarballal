\section{Plain Old Java Object - POJO}

Un \textit{POJO} (acr�nimo de \textit{Plain Old Java Object}) es una sigla creada por Martin Fowler, Rebecca Parsons y Josh MacKenzie en septiembre de 2000 y utilizada por programadores Java para enfatizar el uso de clases simples y que no dependen de un framework en especial. Este acr�nimo surge como una reacci�n en el mundo Java a los frameworks cada vez m�s complejos, y que requieren un complicado andamiaje que esconde el problema que realmente se est� modelando. En particular surge en oposici�n al modelo planteado por los est�ndares EJB anteriores al 3.0, en los que los \textit{Enterprise JavaBeans} deb�an implementar interfaces especiales.

POJO es una nueva palabra para designar algo viejo. No existe en Java una nueva tecnolog�a con ese nombre, sino que el nombre existe en el marco de una revalorizaci�n de la programaci�n \textit{simplemente orientada a objetos}. Esta revalorizaci�n tiene que ver tambi�n con el �xito de lenguajes orientados a objetos m�s puros y sencillos, que empezaron a tomar parte del mercado al que Java apunta, como Ruby y Python.

Durante los ultimos anos, algunas de las APIs de Java EE han sido muy criticadas, especialmente las APIs de mas alto nivel, y en particular, EJB (capa modelo) y JSF (interfaz Web) por las siguientes razones:
\begin{itemize}
	\item Dificiles de usar
	\item No siempre representan \textit{las mejores ideas} sobre como hacer las cosas
	\item Las APIs estandares no siempre tienen todo lo que el desarrollador necesita
	\item Las mejoras tardan en llegar al desarrollador final
	\begin{itemize}
		\item Las nuevas versiones de las APIs tardan en estandarizarse
		\item Y despues hay que esperar a que haya implementaciones robustas
	\end{itemize}
\end{itemize}

\subsection{El concepto de POJO}
El \textit{POJO} o \textit{Plain Old Java Object}, es un concepto que se comenz� a utilizar con los frameworks \textit{no intrusivos} como Spring e Hibernate. La idea del framework \textit{no intrusivo} es que la jerarqu�a de clases del dominio l�gico de la aplicaci�n con el objetivo de que sea m�s escalable y mantenible. Esto es algo muy dificil de lograr, y muy pocas veces la aplicaci�n queda libre de estas dependencias. El antipatr�n por excelencia de la \textit{no intrusi�n} es EJB antes de la versi�n 3. En versiones anteriores a EJB 3.0, las \textit{entidades} y los \textit{session beans} obligatoriamente deb�an implementar y heredar ciertas interfaces y clases prove�das por el framework.

Entonces el POJO es una clase de tu dominio l�gico que esta completamente limpia, en teor�a, de la dependencia de las clases del framework. Eso quiere decir que un POJO hereda de clases internas de la aplicaci�n, y que la clase m�s ancestral de tu aplicaci�n, solo podr� heredar de Object o de alguna de las clases m�s primitivas del JSE. Por otra parte, para la composici�n p�blica con colecciones, s�lo se pueden utilizar la API Collections, o bien alg�n otro POJO de tu aplicaci�n, aunque esto no siempre se cumple. 

En la actualidad podemos destacar los siguiente \textit{frameworks POJO Open Source} para diferentes ambitos.

\begin{itemize}
	\item Capa Modelo
		\begin{itemize}
			\item Hibernate
			\item iBatis
			\item Spring
		\end{itemize}
	\item Capa Vista o Interfaz
		\begin{itemize}
			\item Orientados a acci�n
				\begin{itemize}
					\item Struts
					\item Spring MVC
				\end{itemize}
			\item Orientados a componentes
				\begin{itemize}
					\item Tapestry
					\item Seam
					\item Wicket
				\end{itemize}
		\end{itemize}
\end{itemize}

Una vez que hemos visto esto, podemos hacernos una idea de como ser� la estructura real de nuestro sistema:

\begin{figure}[htb]
\centering
\includegraphics*[width=12cm]{imagenes/pojo.png}
\caption{Estructura sistema POJO}
\label{pojo}
\end{figure}
