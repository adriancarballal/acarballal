\chapter{Pruebas}
\label{pruebas}

Toda aplicaci�n con un tama�o razonable necesita la realizaci�n de pruebas paulatinas a largo de toda su implementaci�n y antes de su implantaci�n. De esta manera, se intentar� en todo momento minimizar el n�mero de errores lo m�s tempranamente posible, para que estos no aparezcan en fases avanzadas del proyecto.
En contreto, durante la realizaci�n de esta aplicaci�n se han utilizado tres tipos de test:
\begin{itemize}
	\item Test de unidad
	\item Test de integraci�n
	\item Test de estr�s
\end{itemize}

A lo largo de este cap�tulo se intentar�n explicar tanto el contenido de cada uno de ellos, como los resultados que se obtenieron en los mismos.
\section{Tests de unidad}
El primer tipo de test que vamos a detallar son los test de unidad. Este tipo de test se utilizan para comprobar y depurar los posibles errores que pudiesen aparecer en la capa modelo. Esta depuraci�n previa permitir� al programador realizar las capas superiores al modelo de una manera mucho m�s fiable.

Para la implementaci�n de este tipo de test se ha aprovechado la automatizaci�n de test que nos proporciona Maven, el cu�l tiene una buena integraci�n con Spring. Maven nos permite realizar estas pruebas mediante el framework JUnit, de modo que Spring pueda inyectar las dependencias de las clases JUnit mediante anotaciones, tal y como se observa a continuaci�n.
\begin{figure}[htb]
\centering
\includegraphics*[width=13cm]{imagenes/junit_test.png}
\caption{Inyecci�n de dependencia en JUnit}
\label{junittest}
\end{figure}

Los test de unidad se han realizado enfocados a cada uno de las fachadas del sistema, cada una de la cuales se corresponde con:
\begin{itemize}
	\item Fachada de usuarios
	\item Fachada de videos
	\item Fachada de administraci�n
\end{itemize}

En la siguiente figura se muestra los resultados de ejecuci�n de los test de unidad obtenidos mediante Maven:
\begin{figure}[htb]
\centering
\includegraphics*[width=13cm]{imagenes/unitTest.png}
\caption{Resultados de los test de unidad}
\label{testunit}
\end{figure}

\section{Tests de integraci�n}
\label{integracion}
\subsection{Portal Web}
\subsection{Portal optimizado para dispositivos m�viles}

\section{Tests de estr�s}
\label{estres}

Se ha creado un escenario de pruebas gen�rico para ambos portales, en el cual se realizan peticiones http sobre los mismo. Estas peticiones en concreto se har�n sobre el portal web ser�n:
\begin{itemize}
	\item Pagina inicial del portal
	\item Busqueda de videos
	\item Bases del concurso 
	\item Mostrar un video concreto
	\item Mostrar comentarios de un video
\end{itemize}
\subsection{Portal Web}
\subsection{Portal optimizado para dispositivos m�viles}
\subsubsection{10 usuarios}


